\documentclass{article}

\usepackage[english, russian]{babel}
\usepackage[letterpaper,top=2cm,bottom=2cm,left=3cm,right=3cm,marginparwidth=1.75cm]{geometry}
\usepackage{listings}
\usepackage{float}
\usepackage{latexsym, amsmath, amssymb}
\usepackage[T2A]{fontenc}
\usepackage[utf8]{inputenc}
\usepackage[russian]{babel}
\usepackage{listings}
\lstset{language=C, extendedchars=\true}

\title{Лабораторная работа №5}
\author{Белецкий\\ Панов}
\date{Декабрь 2022}

\begin{document}
\maketitle

\section{Описание}
Преобразование синтаксиса, записанного в РБНФ в КС-грамматику

\section{Запуск программы}

\begin{enumerate}
  \item Необходимо клонировать репозиторий проекта при помощи команды
  \begin{lstlisting}
    git clone
    \end{lstlisting}
  \item В папке src находится файл main.kt с main функцией программы. Через файл main.kt проиходит запуск программы.
\end{enumerate}

\section{Настройки файла syntax.txt}
Перед вводом данных необходимо внести изменения в файл `syntax.txt`, лежащий внутри папки проекта.
Внутри файла можно задавать конфигурацию следующих параметров:
\begin{itemize}
  \item NStart - символ начала нетерминала. Значение по умолчанию: "\_"
  \item NEnd - символ конца нетерминала. Значение по умолчанию: "\_"
  \item `Arrow` - строка-разделитель для имени нетерминала и его значения. Значение по умолчанию: "::="
  \item `Epsilon` - символ пустого символа. Значение по умолчанию: "\$"
  \item IterStart - символ начала итерации. Значение по умолчанию: "\{"
  \item IterEnd - символ конца итерации. Значение по умолчанию: "\}"
  \item OptionalEnterStart - символ начала условного вхождения. Значение по умолчанию: "[" (Открывающая квадратная скобка)
  \item OptionalEnterEnd - символ конца условного вхождения. Значение по умолчанию: "]" (закрывающая квадратная скобка)
  \item NecessarilyStart - символ начала группировки элементов для обязательного вхождения. Значение по умолчанию: "("
  \item NecessarilyEnd - символ конца группировки элементов для обязательного вхождения. Значение по умолчанию: ")"
  \item Alternative - символ-разделитель правил или "альтернатива". Значение по умолчанию: "\|"

\end{itemize}

Правила задания параметров:

\begin{enumerate}
  \item Каждый новый параметр начинается с новой строчки. Значение задается через символ `=`. Пример задания параметра:
  \begin{lstlisting}
    Arrow = ::=
    \end{lstlisting}
  \item Если в файле указать несуществующий параметр, то он проигнорируется. Пример несуществующего параметра:
  \begin{lstlisting}
    MY_CUSTOM_PARAM = @
    \end{lstlisting}
  \item Если неправильно записать параметр, в выводе появится соответствующая строка, для параметра будет использоваться значение по умолчанию.
Пример сообщения об ошибке:
    \begin{lstlisting}
    Error while parsing string: "Arrow != -->"
    \end{lstlisting}
\item Если не указать какой-либо существующий параметр, выберется значение по умолчанию.
\item Имена различных параметров не должны совпадать, за исключением тех, что представляют собой, например, открывающий и закрывающий символы.
Так параметры NStart и NEnd могут совпадать, а NStart и NecessarilyStart нет.
\end{enumerate}

\section{Тесты}
Вместе с файлами программы в проекте лежат файлы тестов в папках testN. Внтури есть необходимые параметры для файла syntax.txt,
вводимые строки, соответствующие параметрам из файла syntax.txt, а также ожидаемый вывод программы при заданных входных данных и параметрах. Использование теста на примере теста из папки test2:
\begin{enumerate}
  \item Скопировать содержимое файла test2/syntax.txt в src/syntax.txt
  \begin{lstlisting}
    DEFAULT_SETTINGS = true
    \end{lstlisting}
  \item В строку ввода ввести содержимое файла input.txt
  \begin{lstlisting}
    _Number_ ::= ["+"|"-"]_NaturalNumber_["."[_NaturalNumber_]][("e"|"E")["+"|"-"]_NaturalNumber_]
    _NaturalNumber_ ::= _Digit_{_Digit_}
    _Digit_ ::= "0"|"1"|"2"|"3"|"4"|"5"|"6"|"7"|"8"|"9"
    \end{lstlisting}
  \item Так как ввод строк завершается после ввода пустой строки, то вводим ее и получаем на выходе преобразование
    \begin{lstlisting}
    [Nonterm0] -> + | - | e
    [Nonterm1] -> [NaturalNumber] | e
    [Nonterm2] -> .[Nonterm1] | e
    [Nonterm3] -> e | E
    [Nonterm4] -> [Nonterm3][Nonterm0][NaturalNumber] | e
    [Number] -> [Nonterm0][NaturalNumber][Nonterm2][Nonterm4]
    [Nonterm5] -> [Digit][Nonterm5] | e
    [NaturalNumber] -> [Digit][Nonterm5]
    [Digit] -> 0 | 1 | 2 | 3 | 4 | 5 | 6 | 7 | 8 | 9
    \end{lstlisting}
\end{enumerate}

Вместе с выводом дополнительно выводится отладочная информация, однако результат выполнения программы находится в самом конце вывода после строки "КС вид:"
\end{document}